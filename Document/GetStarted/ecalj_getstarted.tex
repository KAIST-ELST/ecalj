\documentclass[a4paper,10pt,epsf,fleqn]{article}
%\documentclass[a4paper]{jsarticle}
\usepackage{graphicx}	% required for `\includegraphics' (yatex added)
\author{Taro Fukazawa, Takao Kotani}
\title{ecalj --- Get statrted ver.0.92}
\begin{document}
\maketitle
\tableofcontents
\newpage
\section{Introduction}
The ecalj is an ``first-principle electronic structure calculation''
package compiled by us. It has kinds of unique features in contrast to other packages.
Especially, we can perform quasiparticle self-consistent GW (QSGW)
calculation based on the PMT method (=LAPW+LMTO method).
ecalj is hosted at \verb+https://github.com/tkotani/ecalj+.
Free to download, modify, and use it, but we ask you to clarify 
using this code in your publications.

In this booklet, we show minimum on the ecalj package.
\begin{itemize}
\item
How to perform self-consistent calculations by the density functional
theory (DF) in the LDA/GGA 
%\footnote{In the followings, DF usually means ``DF in LDA or GGA''}
\item
How to perform the QSGW calculations, and\\ 
\item
how to plot energy bands (BAND), total density of states
(DOS), and the partial density of states (PDOS).
\end{itemize}

To perform calculations are not so difficult.
After we prepare a crystal structure 
(kinds of samples, and a converter from/to VASP-POSCAR format is included.)
Default setting allows you to obtain a reasonable results.
However, for publications, we may need to know a littlle more;
some knowledge to judge whether obtained results 
are converged and reasonable well. 
(Especially, results by QSGW should be examined carefully). 

Let us show features of the ecalj package.

The central part in any electronic structure package is the
``how to solve one-body problem'', that is,
to calculation eigenvalues, eigenfunctions, and the total energy
on the density functional in DFT.
With this as a basis, we can calculate kinds of physical properties.
Or we can do further high-level approximations
such as the 'quasiparticle self-consistent method (QSGW)' as follows.

As a method to treat the one-body problem, ecalj uses the PMT method, which is an
all-electron full potential method where we use not only the
augmented plane waves (APW), but also the muffintion oribtails (MTO),
in addition to the local orbital (lo).
Thus eigenfuncions are represented by linear combinaitons of the
APWs, MTOs, and the lo's.

The ecalj can perform the GW calculation. 
The usual GW approximation is a method to calculate
one-particle energies (eigenvalues). Usually it is done as a
first-order iteration starting from LDA, however it is not necessarily
so good when eigenfunctions and eigenvalues given by LDA are not
so good. Thus T.Kotani and collabolators developed the 
quasiparticle self-consistent GW method (QSGW). The QSGW method is
implementd in ecalj. Default setting allows us to perform QSGW easily.

The QSGW is now getting to be very popular. In principle,
result do not depent on LDA anymore; the LDA results is just used to
prepare initial guess to attain
self-consistency in the QSGW calculaiton. (exactly speaking, we use
LDA as a help in the QSGW code). 
The QSGW calculation requies much computational time 
(roughly speaking, 2 to 20 times more than usual GW).
However, result of QSGW is not only better than GW, but also easier to handle.
Usually the one-particle energies (eigenvalues) obtained by QSGW
reproduce experiments much better than LDA/GGA.
For example, the band gap by GGA for GaAs is about 0.5 eV in contrast
to the experimental value of 1.6eV (if we redo electron-phonon
renomalization from it, the value should be about 1.7eV). 
On the other hand, the QSGW predict about 1.9eV, just slightly larger
than experiment (for pactical use, we sometimes add 
``scaling correction'' on it to have better agreements with experiments).

However, computational costs of QSGW is the main bottle neck. 
Roughly speaking, we can somehow treat ten atom in a cell, and the compuational time
is about 100 to 1000 times more than usual LDA calculations.
(exactly speaking, heavy atoms require longer computational efforts,
light atoms faster).

These PMT and the QSGW are the main feature in the ecalj package.
The ecalj can do LDA+U, relaxiation, spin fluctuation, optical response and so on.
But, in anyway, the main advantage is in the QSGW method; Our recent
development  allows us to do it on the PMT method. 
This is really new, and not yet in a paper (we are going to write a
paper).

If you have some ideas to use this QSGW, let us make
collabolations. Ask to T.Kotani.

The main ecalj web site, and the side of the ecalj package are
\begin{verbatim}
    http://pmt.sakura.ne.jp/wiki/
    https://github.com/tkotani/ecalj
\end{verbatim}
The web site is mainly in Japanese and not well-organized yet (sorry
about it). ecalj is related to lmf7 package compliled by Mark van Schilfgaarde\\
\verb+http://titus.phy.qub.ac.uk/packages/LMTO/fp.html+
. lmf7 and ecalj are branched off about four years ago.
Thus a manual on lmf7 (on the web page, and contained in 
(\verb+ecalj/Dcument/BACKUP/MarksoriginalDoc/+)
are useful to learn ecalj. However, there are differences;
we have not detailed it yet. Sorry about this. 
(we will write a complete ecalj manual soon, but not yet now...)
After branched, major contributions are due to Takao Kotani and Hiori Kino (NIMS).

In anyway, ask me something if you have questions. 
tkotani@damp.tottori-u.ac.jp. Use ecalj in your own resposibility.
When you publish something obtained by ecalj, show it clearly.


%%%%%%%%%%%%%%%%%%%%%%%%%%%%%%%%%%%%%%%%%%%%%%%%%%%%%%%%%%%%%%%%%%%%%%%%
\section{Install}
\label{install}
Look into ecalj/README, or it is shown at https://github.com/tkotani/ecalj.
The installation is not do difficult (especially for gfortran, and
ifort). We have automatic install checker ``make check''.


%%%%%%%%%%%%%%%%%%%%%%%%%%%%%%%%%%%%%%%%%%%%%%%%%%%%%%%%%%%%%%%%%%%%%%%%%%%55
\section{LDA calculations with lm7K}
From codes in lm7K/ directory, we have generated lmf, lmfa and so on
to do LDA calculations. We explain how to do it.

Computational procedure step by step are:
\begin{itemize}
\item Write ctrls by hand, or convert it from POSCAR (in vasp). There
  is a tool to convert POSCAR to ctrls. (ecalj/StructureTool/README).
\item Generate ctrl from ctrls by a script. Then edit ctrl if necessary.
\item Do lmfa, and lmf successively. Then we get self-consistent result of LDA.
\item Plot enery band, DOS, PDOS, by runnning scripts.
\end{itemize}

\subsection{Crystal structure file;ctrls}
\label{ctrls}
Let us explain how to write ctrls.
It might be better to move to a directory for your test, 
e.g, move to \verb+~/ecalj_tutorial+.
\begin{verbatim}
    $ mkdir ~/ecalj_tutorial
    $ cd ~/ecalj_tutorial
\end{verbatim}
(note that \~ means your root(home) directory).

Let us start from "fcc Cu". So, make a directory under \verb+~/ecalj_tutorial+
\begin{verbatim}
    $ mkdir Cu
    $ cd Cu
\end{verbatim}


\paragraph{ctrls}
We already have a sample of Cu in ~/ecalj/lm7K/TESTsamples/Cu/,
where we have a \verb+ctrls.cu+; it is
\begin{verbatim}
    $ cat ~/ecalj/lm7K/TESTsamples/Cu/ctrls.cu
    % const da=0 alat=6.798
    STRUC   ALAT={alat} DALAT={da}
            PLAT=  0.0 0.5 0.5  0.5 0.0 0.5   0.5 0.5 0.0
    SITE    ATOM=Cu POS=0 0 0
\end{verbatim}
The ctrls is somehow flexible to write than POSCAR, and other structure files
in other package.

With lines statring from \verb+%+, we can set values to variables
(in this case \verb+da+ and \verb+alat+). 
The variable can be refereed as \verb+{alat}+; then \verb+{alat}+ is
replaced by the number 6.798.

The lines starting from '\#' is neglected.

Lines without "\#" nor "\%" are the main content in the ctrls file.

The ctrls file must contain two category "STRUC" and "SITE". 
These tags should start from the begining of lines.

Under "STRUC", we have ALAT, DALAT, and PLAT.
ALAT+DALAT specify a unit to mesure length in this ctrl file.
Since DALAT=0 in this example, you can forget it in this example.
Note ALAT is in a.u. (= bohr radius=0.529177\AA). We use this unit
to specify primitive vectors to specify unit cell.
The unit cell is given by PLAT.
In this example three primitive vectors specified by PLAT are
$\vec{a}$=(0,0,0.5), $\vec{b}$=(0.5, 0.0, 0.5), and $\vec{c}$=(0.5, 0.5, 0). 
These are given successively. These are for fcc structure.
DALAT is convenient to change lattice constant; but it is fixed to be
zero here. Not meaningful in this example.

Next, put atoms in the cell.
Set atomic symbole on \verb+ATOM+, and next \verb+POS+
is position in the Cartesian coodinate.
Note the unit is in ALAT+DALAT even here.

copy this ctrls to your test directory.
\begin{verbatim}
    $ cp ~/ecalj/lm7K/TESTsamples/Cu/ctrls.cu .
\end{verbatim}

If you have VESTA, you can see its structure by 
\begin{verbatim}
  $ viewvesta ctrls.cu
\end{verbatim}
(in advance, you have to install it; follow ecalj/Structure/README).

\verb+viewvesta+(\verb+~/ecalj/StructureTool/viewvesta.py+)
makes \verb+POSCAR_cu.vasp+ first, then send it to VESTA.
\verb+viewvesta+ also accept \verb+POSCAR_cu.vasp+ directly.
(Except a file name starting from \verb+ctrl+, \verb+ctrls+,
\verb+viewvesta+ send the file name to VESTA directly).

We have sample in \verb+~/ecalj/StructureTool/sample+.
In addition, a tool \verb+vasp2ctrl+.
This convert POSCAR to ctrls.

\footnote{ ecalj/StructureTool/ is not tested well. Be careful.}

Here is another exmaple with multiple atoms. For example,
\verb+~/ecalj/lm7K/TESTsamples/SrTiO3/ctrls.srtio3+ contains
\verb+SITE+ categoly as
\begin{verbatim}
    SITE
      ATOM=Sr POS=1/2 1/2 1/2
      ATOM=Ti POS= 0   0   0
      ATOM=O  POS=1/2  0   0
      ATOM=O  POS= 0  1/2  0
      ATOM=O  POS= 0   0  1/2
\end{verbatim}
. Note that it is given as 1/2 for POS.
As it show, we can used equations instead of values.
(instead of $3**2$, use $3^2$. You can use parenthese, but not with space
for each.)。

The atomic symbols can be checked by \verb+ctrlgen2.py --atomlist+.

Instead of atomic symbols we can use your own symbol as
\begin{verbatim}
    SITE
      ATOM=M1 POS=1/2 1/2 1/2
      ATOM=M2 POS= 0   0   0
      ATOM=O  POS=1/2  0   0
      ATOM=O  POS= 0  1/2  0
      ATOM=O  POS= 0   0  1/2
    SPEC
      ATOM=M1 Z=38
      ATOM=M2 Z=22
      ATOM=O  Z=8
\end{verbatim}
. The we add extra categoly as SPEC where we set Z number.
(You can use Z=37.5 for virtual crystal approximation, however, 
you can not do it in ctrls. Edit it in ctrl file.)


\paragraph{ctrlgen2.py}
To run programs of lm7K (lmfa and lmf),
you need an input file ctrl which contains many other categories.
However it is time-consuming to do it by hand.
Thus we have "ctrlgen2.py" which generates a default ctrl file from
the ctrls file. Two points. (1)we give reasonable options to invoke
ctrlgen2.py. (2)we need to edit ctrl file afterwards to fit your calculation.

Let us give a point. The names of ctrl file and ctrls must have a name as
\begin{verbatim}
    ctrls.(id)
\end{verbatim}
such as \verb+ctrls.cu+. The ctrl file as well; ctrl.cu.


Try \verb+ctrlgen2.py+ without arguments. It shows help. 

To generate \verb+ctrl+ from \verb+ctrl+, type
\begin{verbatim}
    $ ctrlgen2.py cu --nk=8
\end{verbatim}
Here --nk=8 is an option.
The --nk= means the number of division of the Brillowin zone for
integration. It means 8x8x8 divition. (If your ctrls is anisotropic,
you may like to use 8x8x4; then you have to edit ctrl afterwards).
Note that we supply only the extension of the file \verb+ctrls.cu+.

\begin{verbatim}
    $ ctrlgen2.py cu --nk=8
     === INPUT arguments (--help gives default values) === 
      --help  Not exist
      --showatomlist  Not exist
      --nspin=1
      --nk=8
      --xcfun=pbe   !(bh,vwn,pbe) 103
      --systype=bulk !(bulk,molecule)
      --insulator  Not exist !(do not set for --systype=molecule)

    ...(中略)...

    OK! A template of ctrl file, ctrlgen2.ctrl.cu, is generated.
\end{verbatim}
At this shows,
options which you speified are shown at the beginig of the cosole output
(in this case --nk=8). Others such as --nspin=1 is the default value.

If you like to perform spin-polarized calculations,
do 
\begin{verbatim}
    ctrlgen2.py cu --nspin=2 --nk=8
\end{verbatim}
(when --nspin=2, you have to set initial condition of size of
magnetic moment at each atoms. For this, you have to edit
a section of ctrls (\verb+MMOM=s p d f ...+) to be like
\verb+MMOM=0 0 2+).

The \verb+ctrlgen2.py+ generates ctrl file names as
\verb+ctrlgen2.ctrl.cu+. To do calculations, 
copy it to ctrl.cu.
\begin{verbatim}
    $ cp ctrlgen2.ctrl.cu ctrl.cu
\end{verbatim}


\paragraph{ctrl file}
Before going ahead, let us look into \verb+ctrl.cu+
\begin{verbatim}
    $ less ctrl.cu
\end{verbatim}
showe its contents.

Although \verb+ctrls+ contained only two categories \verb+STRUC+ and
\verb+SITE+ (and \verb+SPEC+), you can seem the file is devided into
many categories such as \verb+VERS+や\verb+IO+. 
The begining of categories are starting from the first column.
Under categories, we have "tokens" such as \verb+ALAT+.
Thus the full name of \verb+ALAT+ can be given as \verb+STRUC_ALAT+.
\verb+SYMGRP+ is a little special category, instead ofgiving values of some tokens under it, we set
generators of space group; when we set \verb+find+,
it automatically calculate symmetry of crystal lattice.
Thus note calculations enforce the symmetry of lattice to be the
symmetry of the electronic structure.

Furthermore, note some special gramatical structure of \verb+SITE+ and
\verb+SPEC+. It contains Z=, POS= and so on under ATOM= token. 
This meand Z= after ATOM= should be taken as sub\_token, that is \verb+SPEC_ATOM_Z+.


Here is a head part of a ctrl file. Pleas look into a a few of ctrl.*
exampls in lm7K/TESTsample/* and CMD22/samples.
\begin{verbatim}
    ### This is generated by ctrlgen2.py from ctrls 
    ### For tokens, See http://titus.phy.qub.ac.uk/packages/LMTO/tokens.html. 
    ### However, lm7K is now a little different from Mark's lmf package in a few poi
    nts.
    ### Do lmf --input to see all effective category and token ###
    ### It will be not so difficult to edit ctrlge.py for your purpose ###
    VERS    LM=7 FP=7        # version check. Fixed.
    IO      SHOW=T VERBOS=35 TIM=2,2
                 # SHOW=T shows readin data (and default setting at the begining of 
    console output)
                 # It is useful to check ctrl is read in correctly or not (equivalen
    t with --show option).
                 # larger VERBOSE gives more detailed console output.
    SYMGRP find   # 'find' evaluate space-group symmetry automatically.
                  # Usually 'find is OK', but lmf may use lower symmetry
    ...(略)...
\end{verbatim}
The ctrl file is an expansion of the ctrls file. 
\verb+#+ means comment lines. We can also use
\verb+%+ means lines to set values for variables.

You can see the original \verb+ctrls+ is embedded in the \verb+ctrl+.
\begin{verbatim}
    ...(略)...
    % const  da=0 alat=6.798
    STRUC   ALAT={alat} DALAT={da}
            PLAT=  0.0 0.5 0.5  0.5 0.0 0.5   0.5 0.5 0.0
      NBAS= 1  NSPEC=1
    SITE    ATOM=Cu POS=0 0 0
    ... (skip) ...
\end{verbatim}

The minimum help of call Category\_token\_subtoken are listed with
minimum explanation with 
\begin{verbatim}
    $ lmf --input
\end{verbatim}
It gives a long output. (Probably it is better to do "lmf -input >lmfinput"
and open lmfinput in an editor to see it).

It shows long output. For example, you can find
\begin{verbatim}
 Token            Input   cast  (size,min)
 ------------------------------------------
    ... ...

     STRUC_ALAT        reqd   r8       1,  1
       Scaling of lattice vectors, in a.u.
    ... ..
\end{verbatim}
This is an minimu explanation of it. "reqd" means "required" (no
default). r8 means it read with real number, 1,1 means
that ALAT=xxx containe one number minimum (max is also one)
(See also STRUC\_PLAT, and so on)

Rather many tokens are common with thoese given in 
\verb+http://titus.phy.qub.ac.uk/packages/LMTO/tokens.html+

Note that ctrl generated by  \verb+ctrlgen2.py+ contains
minimum tokens in detail with explanations.

The crystal structure you supplied can be  checked by 
\verb+lmchk+ 
\begin{verbatim}
    $ lmchk cu
\end{verbatim}
It shows messeges on the crystal structure.

Try \verb+lmchk --pr60 cu+. Then is shows
all the elements of space groups in a simplifeid notation.
See \verb+http://titus.phy.qub.ac.uk/packages/LMTO/tokens.html#SYMGRPcat+ for
the notation. You can choose a few of them in SYMOPS categry if you
restrict symmetry for calculations.

It is possible to deform unit cell by adding optional some tokens at
\verb+STRUC+ categoly. See "STRUC category tokens"; it hits on google.

The SPEC category is like
\begin{verbatim}
SPEC                                                            
    ATOM=Fe Z=26 R=1.70 PZ=0,3.9,3.9  P=0,0,4.2               
      EH=-1 -1 -1 -1  RSMH=0.85 0.85 0.85 0.85          
      EH2=-2 -2 -2   RSMH2=0.85 0.85 0.85
      KMXA={kmxa}  LMX=3 LMXA=4 NMCORE=1                        
      MMOM=0 0 2 0                                                    
  
    ATOM=... (then the similar block of ATOM= are repeated.)
       ...
\end{verbatim}
This is a key point to specify used MTO basis 
located at nucleus as center. 

In this example, we specify MTO besis for the species named as
\verb+ATOM=Fe+. It does not mean all Fe placed in the cell should be
equivalent in space group operation. The equivalence is automatically
judged by the program. Hoewver, you intentionally give different names
for inequivalent atoms. For example, AntiFerro-II NiO conatains 
two NiO in a primitive cell. The it is reaconable to name these two
as Ni1 and Ni2 (Make two ATOM= tokens), although subtokens afterm
ATOM=Ni1 and ATOM=Ni2 are the same.

subtokens Z= is the atomin numbr and R= MT radius.
Note that Fe is just a name to distinguish a sphere in the cell.
If you set SPEC\_ATOM\_Z=27, it is recognazed as Co (since Z=27). 

SPEC\_ATOM\_PZ, SPEC\_ATOM\_EH, SPEC\_ATOM\_RSMH are explained not 
here. Here is minum.
PZ and P are for local orbital basis function.
EH an RSMH are to spcify the 1st set of  MTO.
EH2 and RSMH are to spcify the 2nd set of  MTO.
All of these are give as PZ=s,p,d,f,g,...
(you can use space of , for delimiter).
In cases such as As, the local orbital given by default ctrl
is responsible of rather deep core, and it is not need to be treated 
as valence electrons. In such a case,
remove the local orbital; they take away not only PZ=0,3.9,3.9, 
but also P=0,0,4.2 together.


\subsection{Do LDA calculations. Get convergence on LDA calculation}
\label{lm7K-scf}
Here we get converged results from the \verb+ctrl+
We have to do iteration up to converged.

At first, we need initial guess of charge density.
It can be given by a super position of atomic charge density.
To obtain the charge density, we solve atoms first. It is by
\begin{verbatim}
    $ lmfa cu | tee llmfa
\end{verbatim}
It takes just a few seconds.
Here tee is a command of linux.
It divide console output (standard output) to a file, and to console.

Here to check a key point, do
\begin{verbatim}
    $ grep conf llmfa
\end{verbatim}
. The you see
\begin{verbatim}
conf:SPEC_ATOM= Cu : --- Table for atomic configuration ---
conf When int(P)z .ne. int(P), Qval: Q for MTOcore(PZ)+MTO(P)
conf:  isp  l  int(P) int(P)z    Qval     Qcore   CoreConf
conf:    1  0       4  4         1.000    6.000 => 1,2,3
conf:    1  1       4  3         6.000    6.000 => 2,
conf:    1  2       4  3        10.000    0.000 => 
conf:    1  3       4  4         0.000    0.000 => 
conf:    1  4       5  5         0.000    0.000 => 
conf:-----------------------------------------------------
\end{verbatim}
This is an initial electron distribution, and how we divide 
core and valence. In this case core charege Qcore are (6 for s channel=1s,2s, 
and 6 for p)). Core is not treated separately from valence electrons
(frozen core approximation; we superpose rigid core density to make
all-electron density). Qval means electrons for each s,p,d channels.
The valence channels are 4s,4p,4d, in addtion to 3p and 3d
(this is shown int(P)z column). As for l=0, int(P)=4 and int(P)z=4
(the same number), then we have no 3s, only 4s).

isp means spin (1 or 2), since --nspin=1 for Cu, no isp=2 exist.
In summay we have 4s,3p,4d,3d,4d,4f,5f as valence. 
This means we use correspoing number of MTOs and local orbitals.
Expeciall, 3d and 3d (int(P)z is less than int(P)) are treated by
local orbital.


Let us start calculation.
xo\begin{verbatim}
    $ lmf cu
\end{verbatim}
It might be better to do
\verb+$ lmf cu | tee llmf+
in order to keep cosole output to llmf.
As it is a iteration calculations, it shows similar output again and again.

You can see output like this;
\begin{verbatim}
    ......
       it  9  of 30    ehf=   -3309.962585   ehk=   -3309.962586
     From last iter    ehf=   -3309.962582   ehk=   -3309.962584
     diffe(q)= -0.000002 (0.000007)    tol= 0.000010 (0.000010)   more=F
    c ehf=-3309.9625846 ehk=-3309.9625863
     >>     25.00   exit  lmfp           25.00
     Exit 0 LMF 
     CPU time:   25.000s     Sat Feb 16 12:08:36 2013   on  
    ......
\end{verbatim}
\verb+it  9  of 30+ meand it stop at 9th iteration, althogh we set
maximum number of iteration 30. Note that this number is 
given by \verb+ITER_NIT=30+ in \verb+ctrl.cu+).
\verb+ehf+ and \verb+ehk+ are the ground state energy in Ry.
They are calculated in da littel different manner. Although
they are different dureing iteration, but it finally get to be the
almost ame number.(but it can be different like 10 micro Ry per atom
even converged. But you don't need o care it so much).
ehk:Hohenberg-Kohn energy, ehf: Harris-foulkner energy.

grep diffe lllmf shows how the changed of total energy (and charges)
during iteration. diffe(q) mean  changes of energy with previous
iteration、q is for electron density difference as well.
See also save.* file, which only show ehk and ehk.

Thus we do have ground state energy.
Although output of lmf is long, most of all are to monitor
convergence.
As long as it converged well, you don't need to look into it in detail.
Eigenvalus are shown as
\begin{verbatim}
 bndfp:  kpt 1 of 4, k=  0.00000  0.00000  0.00000   ndimh = 122
 -1.2755 -1.2008 -1.2008 -0.2052 -0.2052 -0.2052 -0.0766 -0.0766 -0.0766
 -0.0174 -0.0174 -0.0174  0.1094  0.1095  0.1095  0.2864  0.2864  0.4170
  0.4170  0.4736  0.6445  0.6445  0.6445
\end{verbatim}
This is at k=  0.00000  0.00000  0.00000 .
(because of histrical reason, two same bndfp: are shown in each iteration).
At, BZWTS: we see the fermi energy (for insulator, we see band gap).
Deep levels less dependent on k are core like levels.


rst.* contains is the main output which contains electron density.
mix.* is a mixing file (which keeps iteration history).
When you restart lmf again, it read rst.cu and mix.cu.
If you start from lmfa result, please remove them.
We can do pararell calculation with lmf-MPIK, 
we can invoke it with mpirun -np 8 lmf-MPIK cu. It should give the
same answer.



%%%%%%%%%%%%%%%%%%%%%%%%%%%%%%%%%%%%%%%%%%%%%%%%%%%%%%%%%%%%
\subsection{DOS plot and Band plot}

We already have script to plot dos and band from the result of lmf
self-consistent calculations.
At \verb+~/ecalj/lm7K/TESTsmaples/*+, we have
\begin{verbatim}
    job_tdos job_band
\end{verbatim}
. Read this script, and then you see how to plot enery bands.


For total DOS plot, we have to modify ctrl file so that
\verb+job_tdos+ add
\begin{verbatim}
          BZ
            SAVDOS=T DOS=-1 1 NPTS=2001
\end{verbatim}
(it is commented out in ctrlgen2.ctrl.cu). 
In addition, we need
\verb+BZ_TETRA=1+(this is defalult; thus make sure that \verb+BZ_TETRA+ do
not exist or \verb+BZ_TETRA=1+). 
In addition, we have to enlarge number of k point \verb+NKABC+
large enough.


\begin{figure}[h]
 \begin{center}
  \includegraphics[width=70mm]{img/ps.dos.cu}
  \vspace{5mm}
  \caption{DOS(Cu)}
 \end{center}
\end{figure}
 

For band plot, we have to set lines along which we plot eigenvalues.
It is given in \verb+syml+. \verb+syml+ also must have extension
as \verb+syml.cu+.

Here let us a sample in syml.cu.
\begin{verbatim}
    $ cp ~/ecalj/lm7K/TESTsamples/Cu/syml.cu .
\end{verbatim}
Thus we have \verb+syml.cu+ to your directory.

Look into \verb+syml.cu+; it is
\begin{verbatim}
    21  .5 .5 .5     0  0 0                L to Gamma
    21   0  0  0     1  0 0                Gamma to X
    0    0 0 0  0 0 0
\end{verbatim}
First line means, we calculate eigenvalues 
for {\bf k} points from {\bf k}=(0.5,0.5,0.5) to {\bf k}=(0,0,0).
"L to Gamma" is just a comment" since program only read seven numbers
for each line.
Second line means, we calculate eigenvalues 
for k points from {\bf k}=(0,0,0) to {\bf k}=(1,0,0).
3rd line means calcultion just stop here.
Units of {\bf k} are in 2$\pi$\verb+ALAT+.

A line starting from '\#' is neglected (comment line).

\begin{figure}[h]
 \begin{center}
  \includegraphics[width=70mm]{img/ps.band.cu}
  \caption{band plot(Cu)}
 \end{center}
\end{figure}

\paragraph{Practice}
Let start from scratch. 
Make directory as \verb+~/ecalj_tutorial/Si+ 
And make \verb+ctrls.si+ first.
Then generates \verb+ctrl.si+, and run lmfa, lmf successively. 
There are other sample. Try it by yourself.

\newpage
\section{QSGW calculation}
\subsection{GWinput}
\label{GWinput}
In order to perform a quasiparticle self-consistent GW (QSGW) calculation, 
\verb+GWinput+ is necessary in addition to \verb+ctrl+.
(All input for start is just ctrl.* and GWinput).
\verb+GWinput+ can be generated automatically by a script
\verb+mkGWIN_lmf2+. 
As in the case of \verb+ctrl+, you may have to edit it in cases.
(especially magnetic moment initial condition "MMOM").

Usage of \verb+mkGWIN_lmf2+: At first you need ctrl file. Then do
\begin{verbatim}
    $ mkGWIN_lmf2 si
    ......
    == Type three integers n1 n2 n3 for Brillowin Zone meshing for GW! ==
     n1=
\end{verbatim}
Then it stops. You have to type three numbers as
2+ return + 2+return+2 return.
\begin{verbatim}
    == Type three integers n1 n2 n3 for Brillowin Zone meshing for GW! ==
     n1= 2
     n2= 2
     n3= 2
    2 2 2
    ...(略)...
    OK! GWinput.tmp is generated!
\end{verbatim}
Generated file is named as \verb+GWinput.tmp+; you have to copy it to \verb+GWinput+.
\begin{verbatim}
    $ cp GWinput.tmp GWinput
\end{verbatim}
These '2 2 2' you typed is reflected in a section 'n1n2n3 2 2 2 ' in \verb+GWinput+.
You can edit it, and change it to e.g. 4x4x4.
In the GWinput file

Here is a part of \verb+GWinput+.
\begin{verbatim}
    $ less GWinput
\end{verbatim}
shows it.
\begin{verbatim}
    ......
    emax_sigm       2.500 !(Ry)  emax cutoff for Sigma
    ......
    <PRODUCT_BASIS> 
     tolerance to remove products due to poor linear-independency
      0.100000D-02 ! =tolopt; larger gives smaller num. of product basis. See lba
    s and lbasC, which are output of hbasfp0.
    ......
\end{verbatim}

In the QSGW calculation, we calculate the self-energy.
Fot this purpose, we have to set some cutoff parameters.

\verb+emax_sigm+ is the maxmum limit of the self-energy (measured from
the Fermi energy). (self-energy shift of higher bands are given in a
simple interolation). I think 2.5Ry is a reasonalbe choice.
If you change it to 3.0Ry to confirm your calculation, 
you have to change \verb+SIGP[MODE=3 EMAX=2.0]+
to \verb+SIGP[MODE=3 EMAX=2.5]+ in \verb+ctrl+ file.
(set EMAX in ctrl a little (0.5Ry) less than \verb+emax_sigm+).

\verb+0.100000D-02 ! =tolopt+gives a number of basis
to expand the Coulomb interaction. (the basis set to expand the
Coulomb interaction and the basis to expand eigenfunction is different).
I think \verb+0.100000D-02+ (=0.001) is small enough (accurate enough).
If you like to reduce computational time use 0.1 or so, but you may
have a little poor results.

After \verb+lcutmx(atom)+, we have arrays of integers, which
corrsponds to cutoff of l for the product basis.

\subsection{do QSGW calculation}
\label{fpgw-calc}
Let us perform QSGW calculation. 
Fot this purpose, we use a script \verb+gwsc+. 
\begin{verbatim}
    gwsc (number of iteration+1) -np (number of nodes) (id of ctrl)
\end{verbatim}
For example, \verb+gwsc 0 -np 6 gas+ is the calculation for 

If (number of iteration+1)=0, it gives one-shot calculation.
But it is different from the usual one-shot in literatures;
since it caltulate off-diagonal elements of self-energy also,
we can plot energy band dispersion plot. 

This is an exaple of one iteration of QSGW cycle.
\begin{verbatim}
    $ gwsc 0 -np 1 si
    ### START gwsc: ITER= 0, MPI size=  1, TARGET= si
    --- No sigm nor sigm.$TARGET files for starting ---
     ---- goto sc calculation with given sigma-vxc --- ix=,0
    No sigm ---> LDA caculation for eigenfunctions 
     OK! qg4gw mode=1 normal mode
     OK! lmfgw mode=1 
     OK! lmf2gw: end --- DATA4GW_V2 is written 
     OK! rdata4gw_v2
     OK! heftet mode=1 EFERMI generated 
     OK! hchknw: write nw to NW
     OK! hbasfp0 ix=3 core mode 
     OK! hvccfp0 imode=3
     OK! hsfp0_sc: Core-exchange mode
     OK! hbasfp0 ix=0 normal mode 
     OK! hvccfp0 imode=0
     OK! hsfp0_sc: Exchange mode
     OK! hx0fp0_sc ixc=11 Sergey F. mode
     OK! hsfp0_sc: Correlation mode
     OK! hqpe_sc 
    0 iteration over
\end{verbatim}
In a machine, (In cases, \verb+STOP+ is shown before \verb+OK!+).

Now get a sigm.si file which contains (static version of) self-energy
minums $V_{\rm xc}^{\rm LDA}$. What we did is the one-shot GW from LDA
result. However, it contains not only diagonal
elements but also off-diagonal elements. We can write energy
dispersion (band plot) from it.
You can make DOS plot and band dispersion,
in the same manner in LDA (you can use 
\verb+job_tdos+ and \verb+job_band+ (but probably need to edit it;
note that syml* is required for band plot).

\begin{figure}[h]
 \begin{center}
  \includegraphics[width=70mm]{img/ps.band.oneshot.si}
  \caption{Si, one-shot GW)}
 \end{center}
\end{figure}

We have to repeat iteration until eigenvalues are converged to get
QSGW results.
Note that total energy given by llmf (also shown in save file) is not meaningful
in the QSGW. Let us repeat 5 iteration more. "-np 1" means one core to use.
\begin{verbatim}
    $ gwsc 5 -np 1 si
    ### START gwsc: ITER= 5, MPI size=  1, TARGET= si
    --- sigm is used. sigm.$TARGET is softlink to it  ---
     ---- goto sc calculation with given sigma-vxc --- ix=,0
    we have sigm already, skip iter=0
     ---- goto sc calculation with given sigma-vxc --- ix=,1
     OK! qg4gw mode=1 normal mode
    ...(略)...
     OK! hsfp0_sc: Correlation mode
     OK! hqpe_sc 
    5 iteration over
\end{verbatim}
Note that we do skip 0th iteration (it means starting from LDA).
Thus we do just five iterations.
Infomation of eigenvalues are in\verb+QPU.(number)run+ files.
(for magnetic systems with nspin=2), wee have \verb+QPD.(number)run+ also).
Check it by ls;
\begin{verbatim}
    $ ls QPU.*run
    QPU0.run  QPU.1run  QPU.2run  QPU.3run  QPU.4run  QPU.5run
\end{verbatim}
These are overwritten when we again repeat gwsc.
Note that \verb+QPU0.run+ was old one when you do 1shot GW from LDA at
the begining.

In order to check the convergece, 
\begin{verbatim}
   $ grep gap llmf*
\end{verbatim}

Let us check convergence of the QSGW calculations.
For this purpose, it is convenient to take a difference of QPU(QPD) files
by a script \verb+dqpu+.
To compare \verb+QPU4.run+ and \verb+QPU5.run+, do
\begin{verbatim}
    $ dqpu QPU.4run QPU.5run
\end{verbatim}
Then we have a list of numbers (these are the differences of values)
shown in both of QPU files). Then it shows
\begin{verbatim}
       Error! Difference>2e-2 between:   QPU.4run   and   QPU.5run  :  sum(abs(QPU-Q
    PU))= 0.05736
       (But small diff can be not a problem. It can be due to a minor machine-depene
    nce.)
\end{verbatim}
However, you don't need to care "Error' so much. You rather need to
check the difference of values.
If not converged well, you may need to repeat 
\verb+gwsc+ again.

\begin{figure}[h]
 \begin{center}
  \includegraphics[width=70mm]{img/ps.band.oneshot.si}
  \caption{band plot(Si, QSGW)}
 \end{center}
\end{figure}


\paragraph{practice}
Perform QSGW calculation for GaAs.
Try one-shot GW, and plot it first.
Then do QSGW. You can try 'n1n2n3 4 4 4', it is a little time
consuming (maybe about 30 minutes per iteration or more).

\section{appendex 1}
\paragraph{ctrls in Si}
\begin{verbatim}
    STRUC   ALAT=10.26
            PLAT=  0.0 0.5 0.5  0.5 0.0 0.5   0.5 0.5 0.0
    SITE    ATOM=Si POS=0 0 0
            ATOM=Si POS=0.25 0.25 0.25
\end{verbatim}


\begin{figure}[hbtp]
 \begin{center}
  \includegraphics[width=70mm]{img/ps.dos.gga.si}
  \vspace{5mm}
  \caption{DOS(Si)}
 \end{center}
\end{figure}


\begin{figure}[hbtp]
 \begin{center}
  \includegraphics[width=70mm]{img/ps.band.gga.si}
  \caption{band(Si)}
 \end{center}
\end{figure}

\begin{figure}[hbtp]
 \begin{center}
  \includegraphics[width=70mm]{img/ps.band.qsgw.gaas}
  \caption{band(GaAs), QSGW (test case))}
 \end{center}
\end{figure}

\section{appendex 2;SO}
\begin{verbatim}
---------------------------
Spin orbit coupling can be added.
Do LDA and/or QSGW with SO=0 first.
Then apply the spin-orbit coupling by perturbation.

After converged with nspin=1 (or 2), 
create new directory and copy
  ctrl.gas, rst.gas, sigm.gas, QGpsi, ESEAVR
to it. Then we set
  nspin=2 
  METAL=3
  SO=1  (this is ldots calculation off-diagonal elements included).
  Q=band (we do not change potential.)
in ctrl.gas. 
Then run
>lmf gas >& llmf_SO
You can see "band gap with SO" by 
> grep gap llmf_SO.
Then you can see two same lines.
 VBmax = 0.101949  CBmin = 0.236351  gap = 0.134402 Ry = 1.82786 eV
 VBmax = 0.101949  CBmin = 0.236351  gap = 0.134402 Ry = 1.82786 eV
(two lines are because of  two-band path mechanism, 
which asks less memory usage than a path mehod)
This is the band gap with SO as a first-order perturbation starting on
top of the "QSGW without SO". When you use ctrl file generated by
ctrlgen2.py. You can do the above procedure with
>lmf --rs=1,0 gas -vnit=1 -vso=1 -vnspin=2 -vmetal=3 --quit=band)
(--rs=1,0 read rst.gas but not write rst.gas. Ses lmf --help
 -vso=1 replace the setting of % const so=0 with so=1).

For band plot, you can use the same procedure 
for the case without SO. (Look into the job_band_nspin1 script.
You have to modify it so that --rs=1,0 gas -vnit=1 -vso=1 -vnspin=2
-vmetal=3 --quit=band is added as arguments for >lmf --band:syml ...).

It is possible to do full self-consisent SO calculations
if Q=band is removed. However, note that Vxc is fixed in QSGW, 
thus less meaningfull (then at least it is better to use FRZWF=T).
--------------------------------------
\end{verbatim}

\section{appendex 3;not described yet...}
\begin{verbatim}
*basis set

*convergence check.

*when calculation fails.
 (1) smaller MT
   (2) fewer PW. smaller pwemax.
   (3) core as semicore.

* conf:

* How to inlcude local orbital?

* meaningless total energy.

* Q0P choice

RSRNGE: enlarge RSRNGE
fpgw/exec/rewriteexe

   Related source codes are in ecalj/lm7K/ .
   A command ecalj/lm7K/ctrlgenM1.py can generate 'standard input file (ctrl file)' 
   just from a given crystal structure file called as ctrls file. 
   Binaries are lmf and lmf-MPIK (MPI k-parelell verion).

   Recently, I renewed some part of algolism of GW/QSGW calculations
   (some ideas are taken from from PRB.81,125102(2010) 
    and Copmuter Physics Comm. 176(2007)1-13).
   ---> this is better than old versions; speed, memory (file size),
   and accuracy for anisortopic systems.
   For comparison, you can use old version in .git (gitk --all and check it out).

In advance, go into ecalj/TOOLS/makedepf90-2.8.8
and do configure and make, then copy makedep to your bin.
(this is a packakge taken from http://personal.inet.fi/private/erikedelmann/makedepf90/.)


-------------------------------------------------
MEMO
-------------------------------------------------
== QSGW for Fe.
  Use 3p as core. Thus need to set PZ=0,3.9 (3p as
  valence). Furthermore, 3d+4d as valence is better 
  (SPEC_ATOM_PZ=0,3.9,4.5 should be added. Need to test more). 
  Use RSRNGE=8 or so (in cases 20 or more required for large systems), 
  for large number of k points.

== Q0P check
   You have to large enough n1n2n3
   It is better to do both of Q0Pchoice=1 and 2.
   (for slabs, Q0Pchoice=2 may be better; need check more. In anyway,
    it is problematic to use unbalanced k points for anisotropic cell).

==EPS mode,
  check Im part of chi0 is smoothly damping at high energy (typically
  1Ry or larger enengy range). If there is some large Im part remains,
  something strange (usually due to orthogonality problem of
  eigenfunctions when you set low q).


==How to set local orbitals.===
  To check used MTO basis, do "lmfa |grep conf". 

  We have to set SPEC_ATOM_PZ= and SPEC_ATOM_P=
  (they ordered as P=s,p,d,f,... )
  
  continious principle quantum number ( 
  P=principleQuantumNumber + 0.5-atan(1/phi dphi/dr) )
  We show two examples here.
 
  When we use lo, we usually take choice 1 or choice 3.
  In such cases, we just need to set PZ (not need to set P).
  PZ for choice 1 is just for initial condition (not affect to final
  PZ for choice 3 may affect slightly to final result).

 Examples:
  Ga p: (in this case, choice 0 or choice 3 is recommended)
     (0)no lo
     (1)3p lo ---> 4p val (when 3p is treated as valence)
     (2)4p lo ---> 5p val (this is for test purpose)
     (3)5p lo ---> 4p val (PZ>P)
     choice 0: 3p core, 4p valence, no lo: default.
          Then we have choice that lo is set to be for 3p,4p,5p.
     choice 1: 3d semi core, 4d valence 
            PZ=0,3.9 (P is not requied. *.9 for core like state)
     choice 2: 4p lo, 5p valence 
            PZ=0,4.5 P=0,5.25
            (NOTE: zero are to use defalut numbers for s,p )
            (NOTE: fractional part .25 is from CPQN for constant
             potential for l=1. The fraction .5 of 4.5 is reasonable for half-filled l).
     choice 3: 4p valence, 5p lo 
            PZ=0,5.25

  Ga d: (in this case, choice 0 or choice 1 is recommended).
     (0)no lo
     (1) 3d lo ---> 4d val  (when 3d is treated as valence)
     (2) 4d lo ---> 5d val  (this is for test purpose)
     (3) 5d lo ---> 4d val  (PZ>P)
     choice 0: 3d core, 4d valecne, no lo: default.
          Then we have choice that lo is set to be for 3d,4d,5d.
     choice 1: 3d lo, 4d valence
              PZ=0,0,3.9  (P is not required. If set, use P=0,0,4.xx;)
     choice 2: 4d lo, 5p valence 
              PZ=0,0,5.15 P=0,0,4.5
               (NOTE: zero are to use defalut numbers for s,p )
               (NOTE: fractional part .15 is from CPQN for constant potential for l=2).
     choice 3: 5d lo, 4d valence,  
              PZ=0,0,5.15

   See pnew section (console output of lmf) for which lmf really use.
   If you like to read from atm.ga file instead of rst file(if exist).
   You have to do lmf --rs=1,1,0,0,1, for example. See lmf --help
   ----------------------------------------
   NOTE: free value for P or PZ.
   free (constant potential fractional number P for
   phi=r^(l+1)). Shown in 'pfree' in console output.
   1.500000
   2.250000
   3.147584
   4.102416
   5.077979
=============================================================

\end{verbatim}

\end{document}
